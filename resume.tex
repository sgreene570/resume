\documentclass[line, margin, 11pt]{res}
\usepackage{hyperref}
\usepackage{textcomp}
\usepackage{tikz}
\usepackage[defblank]{paralist}
\usepackage[none]{hyphenat}
\usepackage{paralist}

\newcommand{\ExternalLink}{%
    \tikz[x=1.2ex, y=1.2ex, baseline=-0.05ex]{%
        \begin{scope}[x=1ex, y=1ex]
            \clip (-0.1,-0.1)
                --++ (-0, 1.2)
                --++ (0.6, 0)
                --++ (0, -0.6)
                --++ (0.6, 0)
                --++ (0, -1);
            \path[draw,
                line width = 0.5,
                rounded corners=0.5]
                (0,0) rectangle (1,1);
        \end{scope}
        \path[draw, line width = 0.5] (0.5, 0.5)
            -- (1, 1);
        \path[draw, line width = 0.5] (0.6, 1)
            -- (1, 1) -- (1, 0.6);
        }
    }

\begin{document}
\name{\Huge{Stephen Greene}}

\begin{resume}

\section{\small OBJECTIVE}
Seeking a Software Engineering role involving Kubernetes and Linux Containers.

\section{\small CONTACT}
sgreene570@gmail.com \\
(908) 300-1030 \\
\url{https://stevegreene.me} \\
\url{https://github.com/sgreene570}

\section{\small EXPERIENCE}
{\bf \large{Associate Software Engineer}} \hfill June 2020 - Present \\
{\bf Red Hat, Raleigh, NC} \hfill \url{https://redhat.com}
\begin{compactitem}
    \item Engineer on the OpenShift Network Edge team.
    \item Work to provide Ingress capabilities to applications running in a managed Kubernetes environment.
    \item Utilize HAProxy and CoreDNS, along with the Kubernetes Operator Pattern, to deliver reliable network configurations for inbound cluster traffic.
    \item Integrate with OpenShift Prometheus components to deliver networking metrics to cluster administrators.
\end{compactitem}


{\bf \large{OpenShift Internship}} \hfill May 2019 - August 2019 \\
{\bf Red Hat, San Francisco, CA} \hfill \url{https://redhat.com}
\begin{compactitem}
    \item Contributed to the OpenShift Machine Config Operator, a Kubernetes Operator that manages cluster nodes.
    \item Developed documentation for Red Hat CoreOS, an immutable operating system intended for use in OpenShift clusters.
\end{compactitem}


{\bf \large{Software Engineering Co-op}} \hfill January 2018 - July 2018 \\
{\bf Intuit, Mountain View, CA} \hfill \url{https://intuit.com}
\begin{compactitem}
    \item Optimized Java backend for Quickbooks Online Accountant.
    \item Leveraged Splunk to create real-time performance metrics.
\end{compactitem}

\section{\small EDUCATION}
{\bf Bachelor of Science in Computer Science} \\
Rochester Institute of Technology, Rochester, NY \\
Attained: May 2020 \\
Minor: Mathematics, GPA: 3.43


\section{\small PROJECTS}
{\bf \ExternalLink \href{https://github.com/sgreene570/audiophiler}{Audiophiler}} \hfill Summer/Fall 2017 \\
Python/Flask web application that leverages AWS S3 and SQLAlchemy to serve audio
files.  Flask API exposes favorite audio files to users through s3 presigned URLs.


\section{\small SKILLS}
\begin{compactdesc}
    \item[Languages] \begin{inparaenum} {Go, C, Python, Java} \end{inparaenum}
    \item[Operating Systems] \begin{inparaenum} {Linux: Debian, Fedora, RHEL, Fedora CoreOS} \end{inparaenum}
    \item[Technologies] \begin{inparaenum} {Kubernetes / OpenShift, Git, GitHub, Vim, Prometheus, Ceph} \end{inparaenum}
\end{compactdesc}

\section{\small ACTIVITIES}
{\bf \large{RIT Computer Science House}} \hfill August 2016 - Present \\
Root Type Person (System Administrator) \hfill \url{https://csh.rit.edu}\\

\end{resume}

\end{document}
