\documentclass[line, margin, 10.5pt]{res}
\usepackage{hyperref}
\usepackage{textcomp}
\usepackage{tikz}
\usepackage[defblank]{paralist}
\usepackage[none]{hyphenat}
\usepackage{paralist}
\usepackage{enumitem}
\setlist[itemize]{leftmargin=*}

\addtolength{\topmargin}{-.25in}

\newcommand{\ExternalLink}{%
    \tikz[x=1.2ex, y=1.2ex, baseline=-0.05ex]{%
        \begin{scope}[x=1ex, y=1ex]
            \clip (-0.1,-0.1)
                --++ (-0, 1.2)
                --++ (0.6, 0)
                --++ (0, -0.6)
                --++ (0.6, 0)
                --++ (0, -1);
            \path[draw,
                line width = 0.5,
                rounded corners=0.5]
                (0,0) rectangle (1,1);
        \end{scope}
        \path[draw, line width = 0.5] (0.5, 0.5)
            -- (1, 1);
        \path[draw, line width = 0.5] (0.6, 1)
            -- (1, 1) -- (1, 0.6);
        }
    }

\begin{document}
\name{\Huge{Steve Greene}}

\begin{resume}

\section{\small MISSION}
Software Engineer specializing in Golang, Kubernetes, and cloud networking, seeking to build scalable and high-performance distributed systems in the RTP area.

\section{\small CONTACT}
sgreene570@gmail.com \hfill (908) 300-1030 \\
\url{https://www.linkedin.com/in/sgreene570} \hfill Raleigh, NC \\
\url{https://github.com/sgreene570}

\section{\small EXPERIENCE}
{\bf \large{Databricks}} \hfill \url{https://www.databricks.com} \\
{\bf Software Engineer (L4)} \hfill June 2023 - Present
\begin{itemize}
    \item Engineer on the Online Tables team alongside former bit.io engineers.
    \item Provide low latency, high throughput access to reverse-ETL data-serving tables.
    \item Enable asynchronous, reliable table creation via durable control plane reconcilers.
    \item Orchestrate decoupled storage and compute machinery via bespoke Kubernetes controllers.
    \item Design and implement partitioned network infrastructure on AWS, Azure, and GCP.
    \item Provide guidance on Kubernetes design, operations, and debugging principles.
\end{itemize}

{\bf \large{bit.io}} \hfill (acquired by Databricks) \\
{\bf Software Engineer (L4)} \hfill July 2021 - June 2023
\begin{itemize}
    \item Engineer for bit.io, an affordable serverless Postgres IaaS startup.
    \item Designed and implemented core Postgres pod-leasing Kubernetes controller.
    \item Optimized cloud resource usage via automated database scaling and cooling.
    \item Deployed database infrastructure via cloud-provider managed Kubernetes clusters.
    \item Maintained developer experience (incl. CI/CD pipelines) to support rapid iteration.
\end{itemize}

{\bf \large{Red Hat}} \hfill \url{https://redhat.com} \\
{\bf Software Engineer} \hfill June 2020 - July 2021
\begin{itemize}
    \item Engineer on the OpenShift Container Platform's Network Edge (ingress) team.
    \item Deployed HAProxy-based ingress controllers via a Kubernetes operator.
    \item Contributed to upstream Kubernetes initiatives, such as CoreDNS \& External-DNS.
\end{itemize}

\section{\small EDUCATION}
{\bf Bachelor of Science in Computer Science} \hfill May 2020\\
Rochester Institute of Technology, Rochester, NY \hfill \url{https://rit.edu} \\
{\bf Activities:\bf} \href{https://csh.rit.edu}{Computer Science House}\\
{\bf Minor:\bf} Mathematics {\bf GPA:\bf} 3.43 (Honors Distinction)

\section{\footnotesize OPEN SOURCE CONTRIBUTIONS}
{\bf \ExternalLink \href{https://openshift.com}{OpenShift}}
    Red Hat's Open Source Kubernetes distribution with extensions.

{\bf \ExternalLink \href{https://github.com/kubernetes-sigs/external-dns}{External-DNS}}
    Kubernetes controller for synchronizing services with DNS providers.

{\bf \ExternalLink \href{https://github.com/kubernetes-sigs/controller-runtime}{Controller Runtime}}
    Framework for developing Kubernetes controllers and operators.

\section{\small SKILLS}
\begin{compactdesc}
\item[Languages] \hfill \begin{inparaenum} {Golang, Scala, Python, PostgreSQL, Rust(learning), \LaTeX} \end{inparaenum}
\item[Operating Systems] \hfill \begin{inparaenum} {Ubuntu, Debian, Fedora, Docker, Podman, MacOS} \end{inparaenum}
\item[Infrastructure] \hfill \begin{inparaenum} {Kubernetes, GCP, AWS, Azure, Postgres, gRPC} \end{inparaenum}
\item[Tools] \hfill \begin{inparaenum} {Git, GitHub, Vim, Jetbrains IDEs, GPG, Helm} \end{inparaenum}
\item[Recreation] \hfill \begin{inparaenum} {Disc golf, Bicycling, Bluegrass music} \end{inparaenum}
\end{compactdesc}

\end{resume}
\end{document}
