\documentclass[line, margin, 11pt]{res}
\usepackage{hyperref}
\usepackage{textcomp}
\usepackage[defblank]{paralist}
\usepackage[none]{hyphenat}

\begin{document}
\name{\Huge{Stephen Greene}}

\begin{resume}

\section {CONTACT}
3018 Nathaniel Rochester Hall \hfill sgreene570@gmail.com \\
Rochester, NY 14623 \hfill (908) 300-1030 \\
\url{https://stevegreene.me} \hfill \url{https://github.com/sgreene570}

\section{EDUCATION}
{\bf Bachelor of Science in Computer Science} \\
Rochester Institute of Technology, Rochester, NY \\
Expected Graduation: Summer 2020 \\
Minor: Mathematics \\
GPA: 3.38

\section{PROJECTS}
{\bf \href{https://github.com/sgreene570/audiophiler}{Audiophiler}} \hfill Summer/Fall 2017 \\
Python/Flask web application that leverages AWS s3 and SQLAlchemy to serve audio
files.  Flask API exposes favorite audio files to users through s3 presigned URLs.
Uses a Jinja2 templated front end to display audio files and allow for web uploading.  SQLAlchemy binds
to a PostgreSQL database to store s3 storage records as well as preffered songs by user.
\\
\\
{\bf \href{https://github.com/sgreene570/SortedDictADT}{SortredDictADT}} \hfill Spring/Summer 2017 \\
Implementation of a sorted dictionary data structure in C.  Uses a dynamically allocated linked list with comparator functions
to recreate the behavior of a sorted dictionary.  Compatible with generic abstract data types.

\section{SKILLS}
\begin{compactdesc}
    \item[Languages] \begin{inparaenum} {Python, Java, C, \LaTeX, Bash, Golang (learning)} \end{inparaenum}
    \item[Operating Systems] \begin{inparaenum} {Linux: Debian, Ubuntu, RHEL} \end{inparaenum}
    \item[Technologies] \begin{inparaenum} {Flask, Ansible, Openshift, PostgreSQL, MySQL, Git, GitHub} \end{inparaenum}
\end{compactdesc}

\section{EXPERIENCE}
{\bf \large{Assistant System Administrator}} \hfill July 2017 - Present \\
{\bf Rochester Institute of Technology} \\
Manage several RHEL and Ubuntu systems using Ansible.  Deploy services on oVirt cluster virtual machines.  Work closely with full
time employees to maintain uptime and meet deadlines.  Use department wide version control to develop automation solutions and write documentation.

\section{ACTIVITIES}
{\bf \large{Computer Science House}} \hfill August 2016 - Present \\
{\bf Root Type Person (System Administrator)} \hfill \url{https://csh.rit.edu} \\
Manage Linux systems for a computing focused living-learning community.
Computer Science House (CSH) has approximately 100 active memebrs and hundreds of alumni
who use house provided services for working on technical projects.  System administrators of CSH
work closely to maximize uptime and maintain the integrity of a class D network.  Ceph, Proxmox,
Openshift, mail, and database services are provided to all members.

\end{resume}

\end{document}

